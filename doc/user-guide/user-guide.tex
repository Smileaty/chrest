\documentclass{article}
\usepackage{fullpage}

\title{CHREST Software: User Guide}
\author{Peter Lane}

\begin{document}
\maketitle

The CHREST shell is an implementation of CHREST with a graphical interface for some common types of models, but also providing a library of functions for developing more sophisticated models.

\section{Graphical shell}

The graphical environment supports the development of simple models, and provides facilities to drive experiments and view information on the resulting model and performance.  The main frame the user sees provides a menu for the 'Data', from which a dataset may be opened.  When a dataset is opened, the main part of the frame will contain a set of controls appropriate to that dataset; examples follow below.  The other menu is for the 'Model', providing access to a view of the model and a dialog to change its properties.  
Figure 1 shows the main components of the model's view.  The current time of the model is shown in the top left corner; two scrollable boxes for the short-term memories show the contents of the nodes referred to; and the right-hand side shows the long-term memory network.  Various parts of the display may be modified: the long-term memory's orientation and size, and resizable bars let you alter the amount of each window that is occupied.  The 'View' menu provides an option to save the long-term memory image to a file.  You may open several view windows onto the same model, and they will all update as the model changes.  This enables you to look at different parts of the long-term memory or details of the short-term memory separately.

The Model/Properties menu option opens a dialog box, shown in Figure 2, to view or change the parameters of the current CHREST model.  
Experimental data and definitions are provided through text files.  The main types are described below.

\section{Learn and recognise}

The basic operations within CHREST are to learn about a new pattern and to retrieve a familiar pattern when given a stimulus.  The learn-and-recognise display allows the user to explore the basic learning mechanisms of the model.  The display is shown in Figure 3.  The list of patterns is read from a data file.  The user highlights a pattern, and then uses one of the buttons on the right either to 'Learn' that pattern, or to 'Recognise' that pattern.  On clicking 'Recognise' the display will show the image of the node retrieved when the highlighted pattern is sorted through the network.  In this case, the pattern <a b> has been retrieved.  To speed up learning, the user can use the 'Learn all' button to learn each pattern once.  
No timing parameters are used in this system, and so it is ideal for observing the basic learning mechanisms within CHREST by keeping a view of the model open to one side. 
An example data file is:
recognise-and-learn 
a b c 
a b 
d e a b

The file starts with the keyword 'recognise-and-learn'.  Each pattern is on its own line.  The atoms are separated by spaces.  The end marker is automatically added to each pattern as it is defined, and so should not be part of the data file.

\section{Paired associate learning}

This form of learning is about learning that one pattern is typically associated with a second pattern: for example, seeing the word 'dog' and associating it with the word 'cat'.  The verbal-learning paradigm was important in psychological research when EPAM was being developed, and the core learning mechanisms of EPAM and subsequently CHREST are based on the empirical support from that time.  The experimental format supported in the interface is the construction of a 'subject protocol', which is a trial-by-trial list of all the responses made by the human participant in the experiment.  For this kind of task, the CHREST model forms sequence links between nodes of the same type in its long-term memory; these links are shown in the model view as a blue number inside the node, the number indicating the linked node.

Figure 4 shows a typical view of the controls for this type of experiment.  The controls on the left display the list of stimulus-response pairs forming one set of data.  The times for which each item in the list is presented, and also the time before the next trial is made, may be altered.  The order of presentation may be as written or random, if the checkbox is ticked.  The  'Run Trial' button will respect the given timings and present each pattern in the list exactly once to the CHREST model.  The model's response to each stimulus is provided in a new column in the protocol display to the right of the display.  A count of the number of errors, responses that are not identical to the target response, is provided below each column.
There are two forms of the verbal-learning experiment.  The first is the paired-associate experiment, where each pair is independent of the other pairs in the list.  The second is the serial-anticipation task, where the idea is to learn the sequence of patterns.  Both tasks can be controlled through the above dialog (although for the second the 'random order' option should not be used).  They have different definition files.
The serial-anticipation experiment is simply a list of patterns, with each item in each pattern separated by a space.  The paired-associate experiment uses a list of pairs of patterns.  Each pair is provided on its own line in the definition file, and the individual patterns of each pair are separated by a colon.

serial-anticipation 
D A G 
B I F 
G I H 
J A L 
M I Q 
P E L 
S U J


paired-associate 
G X J : W A P 
Z X K : S O K 
G X K : Q I L 
L Q F : D A G 
G X F : B I F 
L X J : R O V 
Z H J : S A J 
M B W : B I P 
G Q K : W E K

\section{Categorisation}

Categorisation is the process of assign labels to patterns.  We typically handle categorisation in CHREST by providing the patterns to be named as visual patterns, and the labels as verbal patterns.  Naming links are formed between the two nodes as they co-occur during training; naming links are shown by displaying the number of the linked node in green.
Figure 5 depicts the categorisation experiment display.  This is very similar to that used in the verbal-learning above.  The second of the two patterns is a verbal pattern, not a visual pattern.  Also, using the tick boxes beside each pattern, the user can select which ones will form part of the training data, and which only used for test.  The protocol is based on the response of the model to each visual pattern, after each training cycle.


The data definition file for a categorisation experiment is very similar to that for the paired-associate task, with the pattern to be named and its name presented on the same line, separated by a colon:
categorisation 
1 1 1 0 : A 
1 0 1 0 : A 
1 0 1 1 : A 
1 1 0 1 : A 
0 1 1 1 : A 
1 1 0 0 : B 
0 1 1 0 : B 
0 0 0 1 : B 
0 0 0 0 : B 
1 1 1 1 : X 
1 0 0 1 : X 
1 0 0 0 : X 
0 1 0 1 : X 
0 1 0 0 : X 
0 0 1 1 : X 
0 0 1 0 : X 

\section{Visual Attention and Recall}

CHREST has been used to model the visual attention processes and memory of experts in domains such as chess.  The CHREST shell supports experiments in chess and related domains. 
Figure 6 shows the dialog that appears for training a model on a set of scenes.  The drop-down box at the top allows selection of the domain: specific domains tailor internal processes within CHREST and add heuristics, such as following lines of attack/defence.  The aim of training is to create a CHREST model of a given size.  The choice of maximum number of training cycles is to place an upper limit on training times.  When the parameters are chosen, the 'train' button should be clicked, and the model will be created.  The graph and progress bar will show progress towards the target number of chunks; the process may be stopped by clicking the 'stop' button.
Once trained, the model's recall performance can be tested.  It is possible to use a separate set of files for recall by opening a new data file with the test files; the model will not be changed.  Figure 7 shows the screen on the 'recall' tab.  At the top, a drop-down list can be used to select the target scene.  The button on the right will cause the model to scan the target scene, and the recalled scene will be shown in the lower image.  Some statistics on performance are also shown.

The data definition file for visual stimuli starts with the label 'visual-search', then the height and width of the patterns separated by a space.  A blank line precedes the definition of the stimuli.  A full stop indicates an empty space.  The following example defines two chess positions:
visual-search
8 8

..r.....
pb...pkp
.p.pq...
n..p..p.
...P....
Q..NP.P.
PP...PBP
.....RK.

r..q.rk.
.p...p..
p.n.p.p.
....Pnbp
P....B..
..NB....
.PP.Q.PP
R....R.K

\section{Scripting}

More complex models require more control over the experimental setup, the number of models to train, etc.  For this, control is needed from a program.  As the CHREST shell is implemented in Java, you can use any language which the java virtual machine supports to develop models with CHREST.  The downloadable version has sample scripts for Lisp, Ruby, Groovy (which is based on Java) and Clojure (which looks rather like Lisp).  You can also use Java itself.  The javadoc documentation is provided for all classes provided within the jchrest library.  
For most small projects, it is easy to create scripts using a reasonable editor and running from the command line.  However, it is also possible to use a development environment such as NetBeans or, for Lisp users, an appropriate editor such as Emacs+SLIME.  For illustration, we show how to set up such an environment for Lisp and Ruby.

\subsection{Development environment for Lisp}

The Lisp environment we recommend uses the java-based editor J and its related lisp implementation ABCL.  The website provides a downloadable file containing all the required files and further instructions on developing and running Chrest models.

\subsection{Development environment for Ruby}

The following steps will let you develop models with jruby:
1. Download and install NetBeans, including support for ruby, from http://netbeans.org/ 
2. Create a new Ruby Project, using 'File/New/Project'.  You will get a dialog box similar to the one below.
3. After clicking 'next' we need to tell NetBeans which ruby platform to use.  You need to select a jruby platform, which should already be installed.

4. The final step in setting up the environment is to include the JChrest library within the project. To do this , go to the 'Java' option in the Properties dialog box shown below. Select 'Add JAR/Folder' and add the jchrest.jar file to the classpath for this project. 

5. The only change you need to make to the ruby scripts is to remove the line 'require "jchrest"'; this is taken care of by adding the jar file to the project.

\section{Source code}

The CHREST shell is implemented in Java.  
You can get the source code from http://github.com/petercrlane/chrest

\section{License}

Open Works License.

\end{document}

